\documentclass[12pt,a4paper]{article}
\usepackage[graphicsx]

\title{
  {\Huge{\textsc{Impress-tutorial}}}\\
  {\Large{Yomari Code Camp 2015}}
}

\author{\small{\textsc{Shubha} \\ Sujit Maharjan (BEX 2nd Year, Student of IOE Pulchowk) \\ Manish Munikar(COM 2nd Year, Student of IOE Pulchowk)}}

\begin{document}
\maketitle
\section{Project Title:: Impress-tutorial}

\section{Project Description}

\subsection{Introduction}
This project, entitled “Impress Tutorial” is a simple tutorial-creating web-app that allows us to create tutorials in a step-by-step way that's pleasing for the eye. It is inspired from prezi.com and Bartek Szopka's impress.js script.

Our app provides you simple forms to fill with tutorial slide contents and converts those contents into a CSS3-powered impressive tutorial where only one step is focused at a time.

\subsection{Statement of Problem}
If you search for tutorial on any subject on the World Wide Web. You will probably encounter a blog post with lots of information in text is thrown at you. This is not particularly a good and effective measure to present information to new-comers due to Human Nature of attention span.

So, we want to make it better. Our tutorial will be displayed on a step by step basis. So you will know exactly what you need to do. A task you will be performing will be thus be divided into multiple steps and thus multiple achievement keeping the viewer interested. 

\section{Objective}

How many times have you searched for a tutorial in any subject and got a very long blog post full of small-sized text that hurts the eye? We basically got frustrated of it. Don't you wish those long hard-to-digest tutorials that you read were available as a clear concise step-to-step slides that's pleasing for the eye?

That's exactly what we're trying to create with our app. We wanted a quick and easy way to make tutorials in such a way that we only need to put the content in for every step (or slide) and then it automatically creates a nice-looking tutorial for us with pleasing views and smooth transistions between steps (or slides).

We got our idea of creating such a thing when we heard about impress.js, a JavaScript library that can be used to create tutorials like we wish, created by Bartek Szopka. It is an Presentation API that requires knowledge of HTML, CSS and JavaScript to use. What we've tried to create in this project is an interface for people to use the power of impress.js without having to know the syntax and details of HTML, CSS, or JavaScript. So, that they could focus on thing that they want to do that is crate a good and fantastic looking tutorial.

The major Objective of this project is to change the way tutorial is written in the world wide web. We try to enhance E-learners experience by presenting the tutorial in step by step fashion such that users dont get overwhelm by the large amount of information but can focus on immediate task he need to achieve. 

\section{Project Methodology}

Our project was written as a webapp in PHP, HTML 5, JavaScript, CSS utilising mysql. This project utilises the new html 5 features of canvas and the javascript tool to create a presentation like format of the tutorial.

We need a web server to execute PHP scripts and mysql to store the data. For that we used Apache's local web server. Apache is a free and open-source web server which is probably the most widely used web server at the moment. But that doesn't mean our app won't run on real web servers. All it needs is an Apache (or any other) web server with PHP and mysql support.

Most of the dynamic web programming was done with PHP. PHP is a server-side web scripting language that can be used to dynamically generate html pages. PHP is also one of the most popular server-side scripting languages right now. We have tried to utilise controller and view framework which we came up against when we were learning about web-developement and have implemented it on best of our knowledge.

Impress.js is a presentation framework based on the power of CSS3 transforms and transitions in modern browsers and inspired by the idea behind prezi.com. Impress.js is the heart of our project. We have used the impress.js Presentaion API to create the slides out of the content provided by the user.

\section{Project Output}

Our Web-Application ultimately outputs a static .html file which with required javascript linkage and css linkage so that it looks like a step by step presentation. This file can be viewed from any devices that has a browser and supports the new features of the HTML and Javascript. Thus it is truly portable. If user want to host the tutorial on their own they can download this static page and host it for free on github or other services.

\section{Remarks}

\subsection{Limitation}

Although we have created a working web app, it is in its infancy and it doesn't even utilize more than half of the power of the impress.js script. Impress.js is a script that can create 3D canvas and create lots of animations, transforms and transitions using the power of CSS3. But we haven't been able to utilize them right now. (Too much animation may not even be good for tutorial)

The tutorial web page generated by our app are not independent. So they may not be able to be viewed offline.

\subsection{Thanks}
We would like to thank to people who have created 

\begin{enumerate}
\item impress.js
\item tinymce
\end{enumerate}

We have utilised their code and without them this project would not have been sucessful.

\subsection{Conclusion}

Although our app may not produce perfect tutorials, the ones produced are it are better to read and look at than many of the eye-hurting blog post tutorials out there. In this way, we try to contribute to the “e-learning” theme.

\end{document}
